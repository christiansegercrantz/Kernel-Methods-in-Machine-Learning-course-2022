\documentclass{article}
\usepackage[margin=3cm]{geometry}
\usepackage[utf8]{inputenc}
\usepackage{amsmath}
\usepackage{amssymb}
\usepackage{float}
\usepackage{enumitem}
\usepackage{graphicx}
\usepackage{caption}
\usepackage{subcaption}
\usepackage{comment}
\usepackage{eurosym}

\graphicspath{ {plots/} }

\title{CS-E4830: Kernel methods of machine learning - Assignment 1}
\author{Christian Segercrantz 481056}


\begin{document}
	\maketitle
	\pagebreak
\section*{Task 1}
\begin{align}
	K_1(x,y) = (\langle x,y \rangle+c)^m
\end{align}
We know that the inner product $\langle x,y \rangle$ is a kernel. By the binomial theorem we get 
\begin{align}
	\sum_{k=0}^{m} {m \choose k} \langle x,y \rangle^{m-k}c^k
\end{align}
Since we know that 
\begin{enumerate}
	\item A conic combination of kernels is a kernel
	\item $c\geq 0$ and ${m \choose k} \geq 0$
	\item The product of a kernel is a kernel
\end{enumerate}
We can see that the polynomial kernel is a conic combination of the inner product to a power and thus a kernel: 
\begin{align}
	&\sum_{k=0}^{m} {m \choose k} \langle x,y \rangle^{m-k}c^k \\
	=& \sum_{k=0}^{m} C \langle x,y \rangle^{m-k}, \qquad C= {m \choose k}c^k 
\end{align}

Additionally the last term, then the exponent is 0, is a constant. Hence we can re-write the sum as a sum of kernal and a constant:

\begin{align}
	K_{sum}(x,y) = \langle\phi(x),\phi(y) \rangle + C_m, \qquad C_m = {m \choose m} c^m = c^m
\end{align}

Hence $K_1$ is a kernel.

\section*{Task 2}
\begin{align}
	h(x) =& \text{sgn}\left( \sum_{i = 1}^{n} \alpha_i k(x,x_i) + b\right), \quad k(x,x_i) = \langle \phi(x), \phi(x_i)\rangle \\
	=& sgn(||\phi(x) - c_-||^2 - ||\phi(x) - c_+||^2 )
\end{align}
\begin{align}
	||\phi(x) - c_-||^2 =& \langle \phi(x) - c_-, \phi(x) - c_- \rangle \\
	=& \langle \phi(x), \phi(x)\rangle -\langle \phi(x), c_-\rangle - \langle \phi(x), c_-\rangle + \langle c_-, c_-\rangle \\
	||\phi(x) - c_+||^2 =& \langle \phi(x) - c_+, \phi(x) - c_+ \rangle \\
	=& \langle \phi(x), \phi(x)\rangle -\langle \phi(x), c_+\rangle - \langle \phi(x), c_+\rangle + \langle c_+, c_+\rangle\\
	||\phi(x) - c_-||^2 - ||\phi(x) - c_+||^2 =& \langle \phi(x), \phi(x)\rangle -\langle \phi(x), c_-\rangle - \langle \phi(x), c_-\rangle + \langle c_-, c_-\rangle\\ & -\langle \phi(x), \phi(x)\rangle +\langle \phi(x), c_+\rangle + \langle \phi(x), c_+\rangle - \langle c_+, c_+\rangle \\
	=& -2\langle \phi(x), c_-\rangle + \langle c_-, c_-\rangle +2\langle \phi(x), c_+\rangle  - \langle c_+, c_+\rangle\\
	=& 2\langle \phi(x), c_+\rangle - 2\langle \phi(x), c_-\rangle + \langle c_-, c_-\rangle - \langle c_+, c_+\rangle \label{eq:2}
\end{align}

Since the sign-function is unaffected by a positive scalar multiplier we will divide the expression with 2.

The definition of $c_-$ and $c_+$ are
\begin{align}
	c_- =& \frac{1}{m_-}\sum_{i\in I^-}\phi(x_i)\\
	c_+ =& \frac{1}{m_+}\sum_{i\in I^+}\phi(x_i)
\end{align}
We can write the above inner products as 
\begin{align}
	\langle \phi(x), c_-\rangle =& \left\langle \phi(x), \frac{1}{m_-}\sum_{i\in I^-}\phi(x_i)\right\rangle\\
	=& \frac{1}{m_-}\sum_{i\in I^-}\left\langle \phi(x), \phi(x_i)\right\rangle\\
	=& \frac{1}{m_-}\sum_{i\in I^-}k(x,x_i)
\end{align}
And similarly we get for the + side
\begin{align}
	\langle \phi(x), c_+\rangle =& \frac{1}{m_+}\sum_{i\in I^+}k(x,x_i)
\end{align}
And for the two components containing only the centroids we get 
\begin{align}
	\langle c_-, c_-\rangle =& \left\langle \frac{1}{m_-}\sum_{i\in I^-}\phi(x_i), \frac{1}{m_-}\sum_{j\in I^-}\phi(x_j)\right\rangle\\
	=& \frac{1}{m_-}\sum_{i\in I^-}\frac{1}{m_-}\sum_{j\in I^-}\left\langle \phi(x_i), \phi(x_j)\right\rangle\\
	=& \frac{1}{m_-^2}\sum_{i\in I^-}\sum_{j\in I^-}k(x_i,x_j)\\
	\langle c_+, c_+\rangle =&\frac{1}{m_+^2}\sum_{i\in I^+}\sum_{j\in I^+}k(x_i,x_j) \\
	b = \frac{1}{2}\langle c_-, c_-\rangle - \frac{1}{2}\langle c_+, c_+\rangle =& \frac{1}{2m_-^2}\sum_{i\in I^-}\sum_{j\in I^-}k(x_i,x_j) - \frac{1}{2m_+^2}\sum_{i\in I^+}\sum_{j\in I^+}k(x_i,x_j)
\end{align}
The remaining terms then become
\begin{align}
	\langle \phi(x), c_+\rangle - \langle \phi(x), c_-\rangle =&\frac{1}{m_+}\sum_{i\in I^+}k(x,x_i) -  \frac{1}{m_-}\sum_{i\in I^-}k(x,x_i)
\end{align}
From which we can see that if $x$ is part of the positive cluster, $\alpha_i$ has to be $\frac{1}{m_+}$ and if part of the negative cluster $-\frac{1}{m_-}$. We can thus write Equation \ref{eq:2} as 
\begin{align}
	\sum_{i=1}^{n}\alpha_i k(x,x_i) + b, \qquad
	\alpha_i = \begin{cases}
		\frac{1}{m_+}, & y = 1\\
		-\frac{1}{m_-}, & y= -1
	\end{cases} \\
	\implies h(x) = \text{sgn}\left( \sum_{i=1}^{n}\alpha_i k(x,x_i) + b\right) 
\end{align}


\section*{Task 3}
\begin{align}
	K_2(x,y) =& \cos(x+y)
\end{align}
We prove that $K_2$ is not a kernel by contradiction. Assume that $K_2$ is a kernel, i.e. it has to be positive definite. This means that any point should fulfill the criteria that 
\begin{align}
	\sum_{i=1}^{n}\sum_{j=1}^{n} \alpha_i\alpha_j k(x_i,x_j) \geq 0
\end{align}
for all $n \geq 1$, $\alpha \in \mathcal{R}^n$, $x \in \mathcal{X}^n$. Let's take the example $\alpha_i = \alpha_j = a \neq 0 $ and $x_i=x_j=2\pi$ This gives us $a^2 \cos(\frac{2\pi}{2}) = a^2\cdot -1 = -a^2$. Since $a$ is real and non-zero $-a^2<0$, which mean that the function is not positive definite.
I.e. $\cos(x+y)$ is not a kernel!

\section*{Task 4}
\begin{align}
	K_3(x,y) = \frac{1}{1-xy}, \quad x,y\in (-1,1)
\end{align}
Since $|xy|<1$,$ \frac{1}{1-xy}$ can be written as the series expansion
\begin{align}
	 \frac{1}{1-xy} = 1 + xy + x^2y^2+x^3y^3...
\end{align}
Since products and conic sums of kernals are kernals and the sum of a kernel and a constant is a kernel as well can we conclude that $K_3(x,y)$ is a kernel.

\end{document}




