\documentclass{article}
\usepackage[margin=3cm]{geometry}
\usepackage[utf8]{inputenc}
\usepackage{amsmath}
\usepackage{amssymb}
\usepackage{float}
\usepackage{enumitem}
\usepackage{graphicx}
\usepackage{caption}
\usepackage{subcaption}
\usepackage{comment}
\usepackage{eurosym}

\graphicspath{ {plots/} }

\title{CS-E4830: Kernel methods of machine learning - Assignment 1}
\author{Christian Segercrantz 481056}


\begin{document}
	\maketitle
	\pagebreak
\section*{Task 1}
\begin{align}
	K_1(x,y) = (\langle x,y \rangle+c)^m
\end{align}
We know that the inner product $\langle x,y \rangle$ is a kernel. By the binomial theorem we get 
\begin{align}
	\sum_{k=0}^{m} {m \choose k} \langle x,y \rangle^{m-k}c^k
\end{align}
Since we know that 
\begin{enumerate}
	\item A conic combination of kernels is a kernel
	\item $c\geq 0$ and ${m \choose k} \geq 0$
	\item The product of a kernel is a kernel
\end{enumerate}
We can see that the polynomial kernel is a conic combination of the inner product to a power and thus a kernel: 
\begin{align}
	&\sum_{k=0}^{m} {m \choose k} \langle x,y \rangle^{m-k}c^k \\
	=& \sum_{k=0}^{m} C \langle x,y \rangle^{m-k}, \qquad C= {m \choose k}c^k 
\end{align}

\section*{Task 2}
\begin{align}
	h(x) =& \text{sgn}\left( \sum_{i = 1}^{n} \alpha_i k(x,x_i) + b\right), \quad k(x,x_i) = \langle \phi(x), \phi(x_i)\rangle \\
	=& \text{sgn}\left( \sum_{i = 1}^{n} \alpha_i \langle \phi(x), \phi(x_i)\rangle + b\right) \\
	=& \text{sgn}\left( \sum_{i = 1}^{n}  \langle \alpha_i\phi(x)+ b, \alpha_i\phi(x_i)+ b\rangle \right)
\end{align}
We note that 
\begin{align}
	||\phi(x) - c_-||^2 &= \langle \phi(x) - c_-, \phi(x) - c_- \rangle \\
	||\phi(x) - c_+||^2 &= \langle \phi(x) - c_+, \phi(x) - c_+ \rangle.
\end{align}

\begin{align}
	||\phi(x) - c_-||^2 =& \langle \phi(x) - c_-, \phi(x) - c_- \rangle \\
	=& \langle \phi(x), \phi(x) \rangle - c_-
\end{align}

\section*{Task 3}
\begin{align}
	K_2(x,y) =& \cos(x+y) = \cos(x)\cos(y) - \sin(x)\sin(y)
\end{align}
The feature map would thus be $\phi(x) = \begin{bmatrix}\cos(x) \\ i \sin(x)\end{bmatrix}$


\end{document}




